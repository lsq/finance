\documentclass{ctexart}
% 中文乱数假文(Lorem ipsum)宏包
\usepackage{zhlipsum}
\usepackage{tabularx}
\usepackage{media9}
% ========自定义列表环境=======计数器
\newcounter{mylist}
% 列表环境定义
\newenvironment{myitemize}{%
  \begin{list}{\hspace{2em}\themylist 、}{%
      \usecounter{mylist}%
      \setlength{\topsep}{0pt}%
      \setlength{\partopsep}{0pt}%
      \setlength{\parsep}{0pt}%
      \setlength{\itemsep}{0pt}%
      \setlength{\leftmargin}{0pt}%
      \setlength{\rightmargin}{0pt}%
      \setlength{\labelsep}{0pt}%
      \setlength{\itemindent}{2.0em}%
    }
  } {\end{list} }
  
\begin{document}
% 默认itemize
\zhlipsum[1]
\begin{itemize}
\item \zhlipsum[1]
\item \zhlipsum[2]
\end{itemize}
% 手动分页
\newpage
% 自定义
\zhlipsum[1]
\begin{myitemize}
\item \zhlipsum[1]  
\item \zhlipsum[2]
\end{myitemize}

\section{表格}

\begin{tabularx}{50mm}{l|X} \hline
年份 & 探测目标 \\ \hline
1959 & 苏联月球3号发回月球背面照片。\\
1964 & 美国水手4号飞往火星。\\ \hline
\end{tabularx}



此处是视频文件
\includemedia[
  width=0.4\linewidth,
  height=0.3\linewidth,
  activate=pageopen,
  addresource=a.MP4,
  flashvars={source=a.MP4}]{}{VPlayer.swf}
 \end{document}
\end{document}