\documentclass{ctexart}
\usepackage{enumitem}
\usepackage[top=1cm,bottom=1cm]{geometry}
\setCJKmainfont{FangSong_GB2312}
\title{\heiti{2019年末存货盘点通知}}
\date{}

\AddEnumerateCounter{\chinese}{\chinese}{}

\begin{document}
\maketitle
%\section{一、	盘点时间}
\begin{enumerate}[label={\chinese*、},labelsep=0pt]
 \item 盘点时间

时间:2019年12月29日
  %\item 格式美观


\item 盘点仓库、仓库负责人及盘点人员

\begin{tabular}{|c|l|l|l|l|}
\hline
\heiti{序号} & \heiti{仓库/库位} & \heiti{仓库/库位负责人} & \heiti{盘点人} & \heiti{监盘人}\\
\hline
1 & 原材料仓 & Gavin & Gavin/Sansan & Kunka/Jenny \\
\hline
2 & 加芯仓 & Jeff & 陈国英/王玉虎 & Kunka \\
\hline
%3 & 成品仓 & 待定 & 待定 & 待定\\
%\hline
\end{tabular}

\item 盘点流程

1.	仓库在12月28日工作结束后所有仓位物料归位,各仓位负责人安排自盘。\\
2.	12月29日盘点前K3系统不允许操作库存,财务人员导出库存数据。\\
3.	盘点前由盘点人员准备好盘点工具、财务人员准备库存数据以便核对数据。\\
4.	实施盘点:由仓库盘点人员清点,财务人员负责监盘,确保物料都经过清点并且数据正确录入,出现漏盘、数据录入错误要重新盘点。\\
5.	盘点数据整理:财务人员整理汇总盘点数据。\\
6.	数据比对:财务人员根据盘点数据与K3系统库存数据比对,比较差异,提供差异数据。\\
7.	仓库根据盘点差异数据对差异部分进行复盘,找出差异原因,出具差异原因报告。\\
8.	财务人员再次对差异原因进行确认,对漏盘进行复盘。\\
9.	财务人员出具盘点报告\\

\flushright{\date{\today}\\深圳市巴达木科技有限公司}
\end{enumerate}

\end{document}
