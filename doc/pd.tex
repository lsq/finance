\documentclass[12pt]{ctexart}
\usepackage{enumitem}
\usepackage{graphicx}
\usepackage[top=1cm,bottom=2cm]{geometry}
\setCJKmainfont{FangSong_GB2312}

\newcounter{thirdlist}
\newenvironment{thirdtitle}[1][\thethirdlist、]{%
\begin{list}{#1}{\usecounter{thirdlist}\topsep=0ex\itemsep=0pt\parsep=0pt\partopsep=0pt%
\itemindent=0em\leftmargin=8mm\rightmargin=0mm}}{\end{list}}
\title{\heiti{2019年12月份盘点报告}}
\date{}

\AddEnumerateCounter{\chinese}{\chinese}{}

\begin{document}
\maketitle
%\section{一、	盘点时间}
\begin{enumerate}[label={\chinese*、},labelsep=0pt]
 \item 盘点时间

时间:2019年12月29日
  %\item 格式美观


\item 盘点仓库、仓库负责人及盘点人员

\begin{tabular}{|c|l|l|l|}
\hline
\heiti{序号} & \heiti{仓库/库位} & \heiti{仓库/库位负责人} & \heiti{盘点人} \\\hline
% \hline
1 & 原材料仓 & Gavin & Gavin/Sansan/Kunka/Jenny\\
\hline
2 & 加芯仓 & Jeff & Jeff/Kunka \\
\hline
% 3 & 成品仓 & 待定 & 待定 & 待定\\
% \hline
\end{tabular}

% \item 盘点流程

% 1.	仓库在12月28日工作结束后所有仓位物料归位,各仓位负责人安排自盘。\\
% 2.	12月29日盘点前K3系统不允许操作库存,财务人员导出库存数据。\\
% 3.	盘点前由盘点人员准备好盘点工具、财务人员准备库存数据以便核对数据。\\
% 4.	实施盘点:由仓库盘点人员清点,财务人员负责监盘,确保物料都经过清点并且数据正确录入,出现漏盘、数据录入错误要重新盘点。\\
% 5.	盘点数据整理:财务人员整理汇总盘点数据。\\
% 6.	数据比对:财务人员根据盘点数据与K3系统库存数据比对,比较差异,提供差异数据。\\
% 7.	仓库根据盘点差异数据对差异部分进行复盘,找出差异原因,出具差异原因报告。\\
% 8.	财务人员再次对差异原因进行确认,对漏盘进行复盘。\\
% 9.	财务人员出具盘点报告\\
\item 盘点差异数据
本次盘点盘亏:48种物料,合计:2129324 PCS\\
盘盈:56种物料,合计:1614050 PCS\\
其中贴片电容、电阻部分差异比较大的为:\\
\includegraphics[height=0.8\linewidth,width=\linewidth]{images/cy}
其他差异部分详见:《2019年12月原材料盘点差异表》\\
% 1、1-DR-02-009 盘亏160K\\
% 2、1-DR-02-011 盘亏10K\\
% 3、1-DR-02-010 盘盈478006\\
% 4、1-DR-04-061 盘亏70K\\
% 5、1-DZ-04-015 盘亏100K\\
% 6、1-DZ-04-002 盘亏20K
\includegraphics[height=0.5\linewidth,width=\linewidth]{images/pd}


\item 本次盘点存在的问题

以下几点为本次盘点中存在的主要问题:
\begin{thirdtitle}
\item 漏盘,少盘\\
漏盘原因一是:备料单数据混乱,备料数据与系统备料清单不一致;\\
原因二,盘点过程中盘点人员没有按盘点表对漏盘数据进行核对,对没有盘到的数据进行复盘。
\item 贴片厂退回未入库,没有严格按退料流程退料,打单入库
\item 无单领料,没有严格执行仓库物料出入库制度,按单作业,无单出
库
\item 一种物料,分两个料号录入系统,前期没有严格区分物料型号。
\item 无料号,发现无料号物料,没有在系统及时地更新物料编码。
\item 盘盈、盘亏\\
这部分主要原因为系统数据未核对准确,另一部分原因为录入系统数据出现错误。
\end{thirdtitle}
\item 针对目前存货管理方法几点建议
\begin{thirdtitle}
	\item 仓库要加强物料安全意识,严格按《物料管理制度》出入库;严禁非相关人员出入仓库,无单领用物料;
	\item 物控要严格按单操作K3系统,对系统数据的准确性负责,对于不一致的物料要与仓库管理人员进行核对;每周定期盘点核对出现不一致的物料。
	\item 盘点人员要提前按物料种类导出并打印《盘点表》,盘点过程中要对没有盘点到的物料主动提出复盘,不得漏盘,少盘,对有盘点中有疑问的物料应及时通知仓库管理人员进行核查,确保盘点工作的准确性。
	\item 对没有料号的物料,要及时通知、添加料号。串料号的要及时调整数据,同一种物料多个料号的要提前导出数据合并后再禁用一个重复料号。
	
\end{thirdtitle}
% \flushright{\date{\today}\\深圳市巴达木科技有限公司}
\end{enumerate}

\end{document}
